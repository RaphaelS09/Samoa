
\section{Numerical approach}
\subsection{Introduction: the non-hydrostatic correction}
%kurze Hinführung, was muss diskretisiert werden, fractional step, rolle der korrekturformeln, algorithmus, integralform?
The basis for the method presented in this paper are the non-hydrostatic shallow water equations as given by:

\begin{align}
\boxed{
{\begin{bmatrix}
h \\
hu\\
hv\\
hw
\end{bmatrix}}_{t}
+
{\begin{bmatrix}
hu \\
hu^2+\frac{1}{2}gh^2\\
huv\\
huw
\end{bmatrix}}_{x}
+
{\begin{bmatrix}
hv \\
huv\\
hv^2+\frac{1}{2}gh^2\\
hvw
\end{bmatrix}}_{y}
=
{\begin{bmatrix}
0 \\
-ghb_x -\left([hQ]_x+[q]_{z=b}b_x \right)\\
-ghb_y -\left([hQ]_y+[q]_{z=b}b_y \right)\\
[q]_{z=b}
\end{bmatrix}_.}}
\label{eq:govern}
\end{align}

Here, $h (x,y)=\eta (x,y) +b(x,y)$ denotes the height of the water column, $\eta (x,y)$ the free surface elevation above the mean sea level, $b (x,y)$ the bathymetry, $(u,v,w)$ the depth-averaged velocity vector, $g$ the gravity of Earth and a subscript the partial derivative with respect to the given coordinate. Further, $q (x,y,z)$ represents the non-hydrostatic portion of the pressure and $Q=\frac{1}{h} \int_{b}^{\eta} q \, dz$ the depth averaged non-hydrostatic pressure. These terms stem from the pressure decomposition
\begin{equation}
p= p_H + q = g* (\eta -z) +q
\end{equation}
into a hydrostatic portion $p_H$ and a non-hydrostatic part $q$.


For a detailed derivation, various sources are available, refer e.g. to \cite{samfass14extension}, \cite{cui} or \cite{fuchs}.  


\subsection{Discretization on Triangular Grids in SAMOA}
%initial attempts: finite differences -> problems with number of unknowns, discontinuity
%picture with examples for assembly of the equations, matrix-free jacobi
%derivation of the element matrix, reference element
%boundary conditions: neumann and dirichlet
%rotation pressure gradients/normals
%computation of the boundary integrals
%interpolation of w (vs. averaging) and q on the nodes->adaptivity
%reduction to poisson equation, stencil, pseudo-symmetry